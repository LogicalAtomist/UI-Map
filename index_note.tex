\begin{center}
\kckaddition{[Original note by Pears and McGuinness.]}
\end{center}

\noindent The translators’ aim has been to include all the more interesting words, and, in each case, either to give all the occurrences of a word, or else to omit only a few unimportant ones. Paragraphs in the preface are referred to as P1, P2, etc. Propositions are indicated by numbers without points \kckaddition{[—the points have been restored for the side-by-side-by-side edition—]}; more than two consecutive propositions, by two numbers joined by an en-rule, as 202–2021.

In the translation it has sometimes been necessary to use different English expressions for the same German expression or the same English expression for different German expressions. The index contains various devices designed to make it an informative guide to the German terminology and, in particular, to draw attention to some important connexions between ideas that are more difficult to bring out in English than in German.

First, when a German expression is of any interest in itself, it is given in brackets after the English expression that translates it, e.g. \textbf{situation} [\textit{Sachlage}]; also, whenever an English expression is used to translate more than one German expression, each of the German expressions is given separately in numbered brackets, and is followed by the list of passages in which it is translated by the English expression, e.g.\ \textbf{reality} 1. [\textit{Realität}], 55561, etc. 2. [\textit{Wirklichkeit}], 206, etc.

Secondly, the German expressions given in this way sometimes have two or more English translations in the text; and when this is so, if the alternative English translations are of interest, they follow the German expression inside the brackets, e.g.\ \textbf{proposition} [\textit{Satz}: law; principle].

The alternative translations recorded by these two devices are sometimes given in an abbreviated way. For a German expression need not actually be translated by the English expressions that it follows or precedes, as it is in the examples above. The relationship may be more complicated. For instance, the German expression may be only part of a phrase that is translated by the English expression, e.g.\ \textbf{stand in a relation to one another}; \textbf{are related}
[\textit{sich verhalten}: stand, how things; state of things].

Thirdly, cross-references have been used to draw attention to other important connexions between ideas, e.g. \textbf{true}, cf.\ correct; right: and \textbf{\textit{a priori}}, cf.\ advance, in.

In subordinate entries and cross-references the catchword is indicated by $\sim$, unless the catchword contains /, in which case the part preceding / is so indicated, e.g.\ \textbf{accident}; $\sim$\textbf{al} for \textbf{accident}; \textbf{accidental}, and \textbf{state of /affairs}; $\sim$ \textbf{things} for \textbf{state of affairs}; \textbf{state of things}. Cross-references relate to the last preceding entry or numbered bracket. When references are given both for a word in its own right and for a phrase containing it, occurrences of the latter are generally not also counted as occurrences of the former, so that both entries should be consulted.
